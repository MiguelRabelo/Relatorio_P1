\section{Estabilidade em Solo}
Visando manter a aeronave em curso na pista, evitando tombamentos, o passeio do cg e os carregamentos dinâmicos sofridos pela mesma. A escolha pela configuração triciclo dá-se em [INTEGRAÇÃO]. O TDP foi dimensionado segundo [GUDMUNDSSON], respeitando dados estruturais, raio de manobra em solo e da velocidade do vento de SJC , obteve-se o valor de 63$^{\circ}$ para o \textit{overturn angle}. Quanto à dirigibilidade foi garantido que o ponto de giro da aeronave se encontrasse dentro da corda da asa .\\
\\
\textcolor{red}{As imagens abaixo são meramente ilustrativas, pois até agora não obtivemos dados das rodas para gerar os dados reais}\\
\\
Modelando a aeronave segundo [Race car vehicle dynamics] longitudinalmente como um biciclo e posteriormente levando em consideração a transferência de carga lateral, obtendo conforme a SAE \textit{Steer Definition}  um veículo com direção \textit{oversteer} e marge estática no solo de \textcolor{red}{XX}.
\\
\textcolor{red}{FIGURAS}\\
\\ 
Apesar do modo de direção, \textit{oversteer}, ser divergente para velocidades acima da velocidade crítica, nesse caso \textcolor{red}{XX}. A aeronave obteve desempenho satisfatório durante o taxiamento e corrida de decolagem