\documentclass[a4paper,12pt]{article}  % Seu arquivo fonte precisa conter %report

\usepackage[a4paper,left=2.75cm,right=2cm,top=2.5cm,bottom=2.0cm]{geometry} %Ajustar margens

%% ADICIONAR PACOTES
\usepackage[brazil]{babel}		% Traduzir para português
\usepackage[utf8]{inputenc} 	% Usar acentos normalmente
\usepackage{indentfirst}		% Dar espaçamento de Parágrafos na primeira linha
\usepackage{pslatex}			% Algum ajuste da fonte
\usepackage{fancyhdr}			% Ajustar cabeçalho e rodapé

\usepackage{dblfloatfix}		%Conseta erros de alinhamento  multi-colunas e subfiguras
%\usepackage{fixltx2e} %Está obsoleto???
\usepackage{multicol}			%Para configurações de multicolunas
\usepackage{scrextend}			%Permite ajustar margens nas figuras
\usepackage{setspace}			% Espaçamento entre linhas
\usepackage{titlesec}			% Espaço entre títulos
\usepackage[tocflat]{tocstyle}
%\usepackage{titletoc}
  
%\usepackage{color}
\usepackage[table,xcdraw]{xcolor}% Adicionar cores: Ao texto, fundo e páginas
\usepackage{colortbl}			 % Ajustar as cores nas tabelas

\usepackage[hidelinks]{hyperref} % Para linkar as referencias
\usepackage[hypcap]{caption} 	 % Para linkar os hiperlinks a lengend
\usepackage[normalem]{ulem} 	 % Mudar enfase: italico ou negrito na referencia
\usepackage{subcaption} 		 % Para colar as sublegendas em floats

\usepackage{float}				 % Costumizar ambientes float
%\usepackage{placeins} %isso é necessário?
\usepackage[pdftex]{graphicx} 	 % Incluir figuras 
\usepackage{longtable} 			 % Para talebas longas, mais de uma página 
\usepackage{tabulary}			 % Ajustar tabelas de acordo com as margens
\usepackage{amsmath} 			 % Para comendos das equações

\usepackage{cite} 				 % Para as referências
%\usepackage{natbib} 			 % Para ajustar as citações
\usepackage{url} 				 % Para citar links
%\usepackage{authordate1-4} 
%\usepackage{times}
\usepackage{verbatim}			 % Ambiente que o latex não entende como latex
\usepackage[alf]{abntex2cite}


%CONFIGURAR ESPAÇAMENTOS
\setlength{\parindent}{28pt}	% Espaçamento de ínicio Paragráfos
\setlength{\parskip}{0pt} %Espaçamento entre paragrafos
%\onehalfspacing %Espaçamento entre linha: 1,5
\doublespacing %espaçamento duplo
\titlespacing{\section}{\parindent}{6pt}{12pt}%{esquerda}{antes}{depois}
\titlespacing{\subsection}{\parindent}{0pt}{0pt}%{esquerda}{antes}{depois}
\titlespacing{\subsubsection}{0pt}{0pt}{0pt}%{esquerda}{antes}{depois}

%CONFIGURAR LEGENDAS
\captionsetup{labelfont={bf, small},textfont={bf, small}, labelsep=endash}	
\numberwithin{equation}{section}
\numberwithin{figure}{subsection}
\numberwithin{table}{subsection}

%CONFIGURAR CORES
\definecolor{tabela_titulo}{HTML}{3CB375}  %{4B0082} 
\definecolor{tabela_par} {HTML}{E3F6CE}    %{HTML}{D8BFD8} 
\definecolor{tabela_impar}{HTML}{FFFFFF}   %{F5FFFA}
\definecolor{destaque}{RGB}{226,239,217} 
\definecolor{texto_titulo}{RGB}{0,0,0}     %{HTML}{F5FFFA}%   

%CONFIGURAR COMANDOS
\newcommand{\Destacarlinha}{\rowcolor{tabela_titulo} \bf}
\newcommand{\Destacar}{\cellcolor{tabela_titulo}\color{texto_titulo} \textbf}
\newcommand{\verde}{\cellcolor{tabela_par}\color{texto_titulo}}
\newcommand{\branco}{\cellcolor{tabela_impar}\color{texto_titulo}}
\newcommand{\Alternarcores}{\rowcolors{1}{tabela_impar}{tabela_par}}

%CONFIGURAR AMBIENTES
\let\oldtable\table
\let\endoldtable\endtable
\renewenvironment{table}{ \oldtable [H] \centering  \Alternarcores}{\endoldtable}

%\let\oldlongtable\longtable
%\let\endoldlongtable\endlongtable
%\renewenvironment{longtable}{ \oldlongtable }{\endoldlongtable}

\let\oldtabulary\tabulary
\let\endoldtabulary\endtabulary
\renewenvironment{tabulary}{\oldtabulary{\textwidth} }{\endoldtabulary}

\let\oldfigure\figure
\let\endoldfigure\endfigure
\renewenvironment{figure}{\oldfigure[H] \centering }{\endoldfigure}

%Ajustar sumário
\usetocstyle{standard} %Negrito nos títulos principais
\setcounter{tocdepth}{2} %0 - part; 1-section;2-subsection;3-subsubsection...
\newcommand{\sumario}{\tableofcontents \thispagestyle{empty} \setcounter{page}{1} \pagebreak} %Ajustar paginação

%CONFIGURAR BIBLIOGRAFIA
%\usepackage{}
%\bibliographystyle[num]{abntex2} %{abstract}

%\begin{comment}
%	ieeetr é numerica, ordem que aparece no texto
%	usar apalike (autor e data) com o cite = [auto, ano], com o natbib = autor (ano) e com a author-date = (autor, ano)
%	abstract com o cite = [referencia], com o author-date = (referencia)
%	obs: colocar \citeauthor para citar FAR ou CS-VLA
%\end{comment}

%% CONFIGURAR CABEÇALHO E RODAPÉ
\pagestyle{fancy} %estilo fancy

\lhead{\begin{addmargin}[-2.5cm]{-3cm}\vspace{-10mm} \includegraphics[height = 20 mm]{CONFIGURAR/logoaero.png} \end{addmargin}}%\includegraphics[scale=0.2]{logoufpe.png}} % esquerda do cabeçalho
%\chead{} %centro do cabeçalho
\rhead{\begin{addmargin}[18.3cm]{-2 cm}\vspace{- 10mm}\includegraphics[height =20 mm]{CONFIGURAR/logoufpe.png}\end{addmargin}}%AERODINÂMICA} % direita do cabeçalho
\lfoot{} %esquerda do rodapé
\cfoot{} %centro do rodapé
\rfoot{\vspace{-1 cm}\thepage} %direita do rodapé
\renewcommand{\headrulewidth}{0pt}  
%\renewcommand{\footrulewidth}{1pt}

%% CONFIGURAR CAPA
%\begin{comment}  CAPA:
\newcommand{\capa}{\thispagestyle{empty}
\begin{figure*}[htp] \begin{addmargin}[-2.75cm]{-2 cm}  \vspace{-2.5cm}
\includegraphics[width= \paperwidth, height = \paperheight]{CONFIGURAR/capa.png}
\end{addmargin} \end{figure*} \clearpage}
